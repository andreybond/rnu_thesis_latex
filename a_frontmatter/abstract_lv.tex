\frontmatterpage
\unchapter{Anotācija}
Bakalaura darba mērķis ir pasākumu izstrāde personāla motivācijas sistēmas uzņēmumā “X” pilnveidošanai, lai mazinātu personāla mainību un ar to saistītās izmaksas. Darbs sastāv no ievada, četrām daļām, secinājumiem, literatūras saraksta un pielikumiem.

Pirmajā daļā analizētas teorētiskās pieejas personāla motivācijas sistēmas izpētei.
Otrajā daļā sniegta uzņēmuma “X” darbības vispārīga analīze.
Trešajā daļā veikts uzņēmuma “X” personāla motivācijas sistēmas pētījums.
Ceturtajā daļā piedāvāti priekšlikumi motivācijas sistēmas pilnveidei un tās ekonomiskās efektivitātes novērtējums.

Metodiskā bāze: zinātniskā literatūra, statistikas dati un uzņēmuma “X” iekšējā dokumentācija; izmantotas salīdzinošās analīzes, statistikas un aptaujas metodes.

\textbf{Atslēgvārdi:} personāls, personāla vadība, personāla motivācija, motivācijas sistēma, materiālā motivācija, nemateriālā motivācija, izmaksas.
