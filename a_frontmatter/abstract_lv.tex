\frontmatterpage
\unchapter{Anotācija}

Anotācija ir īss darba kopsavilkums. Tās mērķis ir sniegt lasītājam pārskatu par izvēlēto tēmu, darba struktūru, izmantotajām metodēm un galvenajiem rezultātiem. Anotācijas apjoms nedrīkst pārsniegt 300 vārdus, un tā jāveido kā viens nepārtraukts teksts (bez aizzīmēm, tabulām, attēliem vai atsaucēm).  

Parasti anotācija sākas ar \textbf{īsu tēmas ievadu}. 1–2 teikumos jānorāda, par ko ir darbs un kāpēc tēma ir aktuāla. Pilnais mērķis, uzdevumi vai hipotēze šeit nav jāatkārto – tie tiek detalizēti atspoguļoti Ievadā.  

Tālāk jāsniedz \textbf{darba struktūras pārskats pa nodaļām}.  
Piemērs:  
"Darbs sastāv no <NODAĻU SKAITS> nodaļām. 1. nodaļā aplūkoti <TEORĒTISKIE PAMATI>. 2. nodaļā sniegta <NOZARES/PROCESA/IZPĒTES OBJEKTA ANALĪZE>. 3. nodaļā veikts <EMPĪRISKAIS PĒTĪJUMS VAI PRAKTISKĀ IZPĒTE>. 4. nodaļā izstrādāti <UZLABOJUMI / MODEĻI / RISINĀJUMI>."  

Pēc tam jāiekļauj īss pārskats par \textbf{izmantotajām metodēm}, bet bez liekas detalizācijas.  
Piemērs:  
"Pētījuma metodoloģiskais pamats ietver zinātniskās literatūras analīzi, statistisko datu izvērtēšanu un citus attiecīgus avotus. Darbā izmantotas šādas metodes: <UZSKAITĪT METODES PIEMĒRAM: salīdzinošā analīze, aptauja, eksperiments, modelēšana>."  

Tālāk 2–3 teikumos jāapraksta \textbf{galvenie rezultāti un secinājumi}, uzsverot tikai būtiskāko ieguldījumu.  

Noslēgumā jānorāda \textbf{darba apjoms}.  
\thesisscopeLV

\textbf{Atslēgvārdi:} jānorāda 5–7 galvenie jēdzieni, kas raksturo darba tēmu un saturu (piemēram, <ATSLĒGVĀRDS1>, <ATSLĒGVĀRDS2>, <ATSLĒGVĀRDS3>).
