% \chapter{Drawing up the List of References}

% The List of References is placed at the end of the thesis. In this template it is managed with \texttt{biblatex} + \texttt{biber}. \emph{Use one citation style consistently across the whole work}: the class defaults to author--year inline citations; if you prefer footnote-style citations, load the class with \texttt{[citestyle=footnote]}.

% \medskip
% This chapter explains how to: (1) add a new citation in the text, (2) add a new source to your bibliography database, (3) find and reuse a source, (4) handle placeholders and page numbers, and (5) generate the final list.

% \section{Adding a New Citation to the Document}\label{sec:add-cite}

% Place the cursor where you want a citation and use \verb|\parencite| (author--year in parentheses) or \verb|\textcite| (author as part of the sentence). You can add page numbers and other locators in optional arguments.

% \paragraph{Inline (author--year) examples}
% \begin{itemize}
%   \item Parenthetical: \verb|\parencite{porter2008}| $\rightarrow$ \parencite{porter2008}.
%   \item With page: \verb|\parencite[p.~34]{porter2008}| $\rightarrow$ \parencite[p.~34]{porter2008}.
%   \item In-text: \verb|\textcite{sedlacek2011}| argues \dots\ $\rightarrow$ \textcite{sedlacek2011} argues \dots
% \end{itemize}

% \paragraph{Footnote-style citations (Option~2)}
% If you loaded the class with \verb|\documentclass[citestyle=footnote]{isma-thesis}|, use \verb|\autocite| for footnotes; e.g., \verb|Sentence.\autocite[pp.~263--278]{lapina2016}|.

% \paragraph{Quotations and ``cf.''}
% Short exact quotes go in \verb|\enquote{...}| (provided by \texttt{csquotes}), followed by a citation: \enquote{Sample quotation} \parencite[p.~11]{sedlacek2011}. When comparing, use \emph{cf.} and cite at the end of the sentence: cf.\ \parencite{porter2008}. Avoid excessive quotation without your own analysis.

% \section{Adding a New Source}\label{sec:add-source}

% All sources live in \texttt{bibliography/references.bib}. Add a new Bib\LaTeX\ entry of the appropriate type (\texttt{@book}, \texttt{@article}, \texttt{@online}, \texttt{@inproceedings}, etc.). Prefer adding a DOI and URL when available.

% \paragraph{Minimal examples}
% \begin{verbatim}
% @book{example-book,
%   author    = {Surname, Name},
%   title     = {Title of the Book},
%   year      = {2020},
%   publisher = {Publisher},
%   address   = {City},
%   doi       = {10.1234/abcd.2020.001},
% }
% @article{example-article,
%   author  = {Surname, Name and Coauthor, Other},
%   title   = {Article Title},
%   journal = {Journal Name},
%   year    = {2021},
%   volume  = {10},
%   number  = {2},
%   pages   = {123--145},
%   doi     = {10.5678/journal.2021.10.2.123}
% }
% @online{example-online,
%   title   = {Official Standard Title},
%   year    = {2015},
%   url     = {https://example.org/standard},
%   urldate = {2025-08-25},
%   note    = {International Organization for Standardization}
% }
% \end{verbatim}

% \paragraph{From Mendeley/Zotero}
% Export your references as \emph{Bib\LaTeX} (not plain BibTeX) and copy them into \texttt{references.bib}. Protect proper nouns that must retain capitalization (e.g., acronyms) using braces: \verb|title = {On {AI} in {HEIs}}|.

% \section{Finding a Source}\label{sec:find-source}

% Reuse existing entries by their citation keys (e.g., \texttt{porter2008}, \texttt{sedlacek2011}). In editors like VS~Code, search within \texttt{references.bib}. If you momentarily need to see all entries in your bibliography, use:
% \begin{verbatim}
% \nocite{*}
% \end{verbatim}
% This includes all database entries in the bibliography (useful for checking; remove before final submission).

% \section{Editing a Citation Placeholder}\label{sec:placeholder}

% If you don't yet have full data, you can:
% \begin{enumerate}
%   \item Insert a temporary citation with a provisional key, e.g., \verb|\parencite{todo-source}|. The compiler will warn you (``\texttt{Citation 'todo-source' undefined}'').
%   \item Or add a stub entry in \texttt{references.bib} with as much information as you have and complete it later.
% \end{enumerate}
% To add page numbers or other locators after you have the source, edit the citation: \verb|\parencite[pp.~94--99]{porter2008}|.

% \section{Drawing up the List}\label{sec:draw-list}

% Insert the bibliography where you want it to appear (typically after \texttt{Conclusions and Proposals}) using:
% \begin{verbatim}
% \printbibliography
% \end{verbatim}
% The class is configured to use single line spacing within each entry and a 6\,pt gap between entries, in line with the guideline's formatting suggestions.

% \paragraph{Build sequence} Compile with:
% \begin{verbatim}
% xelatex main
% biber  main
% xelatex main
% xelatex main
% \end{verbatim}

% \paragraph{Sorting and language notes}
% Entries are sorted by name, year, title (\texttt{sorting=nyt}). Ensure your non-Latin sources have correct transliteration when needed. Provide total page counts in book entries where applicable. When citing a specific page in text, always include the page locator (e.g., \verb|[p.~508]|).

% \paragraph{Consistency rule}
% Choose either inline (author--year) or footnote citations and \emph{use only that form} throughout the thesis, as required by the guideline. The template supports both via a class option (see Section~\ref{sec:add-cite}).

% \section*{Examples in Context}
% The following sentences illustrate typical usage:\par
% The concept is widely discussed \parencite[p.~34]{porter2008}; see also \textcite{sedlacek2011} for a philosophical perspective. For quality in higher education, compare \parencite{lapina2016}.


\chapter{Drawing up the List of References (LaTeX Edition)}

This chapter explains how to manage citations and the bibliography in \LaTeX{} using \texttt{biblatex} (with the \texttt{biber} backend). The template already loads \texttt{biblatex} and adds \texttt{bibliography/references.bib}.

\section{Choose a Consistent Citation Mode}
Two options are supported; pick \emph{one} for the whole thesis:
\begin{enumerate}
  \item \textbf{Author–year (in-text)} — typical for social sciences. Load the class normally:
  \verb|\documentclass{isma-thesis}| (default \texttt{authoryear} style).
  \item \textbf{Footnote citations} — details in footnotes. Load:
  \verb|\documentclass[citestyle=footnote]{isma-thesis}|.
\end{enumerate}

\noindent The final bibliography (List of References) is always typeset with \verb|\printbibliography|.

\section{Add a New Source}
\begin{enumerate}
  \item Open \texttt{bibliography/references.bib}.
  \item Add a Bib\LaTeX{} entry (see §\ref{sec:examples} for all types).
  \item In your text, cite it: e.g., \verb|\parencite{porter2008}| or \verb|\textcite{porter2008}|.
  \item Rebuild: \texttt{xelatex → biber → xelatex → xelatex}.
\end{enumerate}

\section{Citing in the Text}
\begin{itemize}
  \item \verb|\parencite{key}| → (Author, Year, p.~xx) if page given.
  \item \verb|\textcite{key}| → Author (Year) ... in prose.
  \item \verb|\parencites{key1}{key2}| → multiple sources separated by semicolons.
  \item \verb|\footcite{key}| (only in footnote mode) → full note in a footnote.
  \item Add page(s): \verb|\parencite[34]{porter2008}| or ranges \verb|\parencite[34–36]{porter2008}|.
\end{itemize}

\section{The List of References}
Place the list where you need it (typically after Conclusions):
\begin{verbatim}
\printbibliography
\end{verbatim}

\subsection*{Latin then Cyrillic (optional)}
If you must print Latin-alphabet sources first and Cyrillic after (per your guidelines), tag entries with \texttt{keywords} \texttt{latin} / \texttt{cyrillic} in the \texttt{.bib} and print two lists:
\begin{verbatim}
\printbibliography[title={List of References (Latin)}, keyword=latin]
\printbibliography[title={List of References (Cyrillic)}, keyword=cyrillic]
\end{verbatim}
(See examples in the provided \texttt{.bib}.)

\section{Non-Latin Sources and Transliteration}
Enter authors/titles in their original script (Unicode). If the source is in a language different from the thesis language and you translate a quotation yourself, add ``author’s translation'' in your running text, not in the bibliography. Transliteration (if needed) can go into the \texttt{translit} or \texttt{note} field.

\section{Standards, Laws, Company Materials, Interviews}
Use the dedicated entry types/fields shown in §\ref{sec:examples}. For laws and regulations, \texttt{@legislation} is recommended; include the official date, URL, and access date. Company reports can be \texttt{@report} with \texttt{institution}. Interviews can be \texttt{@misc} with \texttt{entrysubtype=interview} and date.

\section{Examples by Source Type}
\label{sec:examples}
The accompanying \texttt{references.bib} contains examples for:
\begin{enumerate}
  \item Books: one author; two–three authors; more than three (``et~al.''); edited volume.
  \item Journal article; book chapter in an edited volume.
  \item Conference proceedings paper.
  \item Internet sources: with named author; without named author (laws/regs, institutional pages); multiple online pieces.
  \item Dictionaries: printed; online headword entry.
  \item Standards: ISO pages (online); standards as a book.
  \item Company materials (internal/annual report).
  \item Theses (Bachelor/Master).
  \item Interview.
  \item Seminar/Conference handout or presentation.
\end{enumerate}

\noindent \textbf{Quick demo:}
\begin{itemize}
  \item Author–year: \verb|\parencite{porter2008}| → (Porter, 2008).
  \item Page pinpoint: \verb|\parencite[508]{kotler2016}| → (Kotler et~al., 2016, p.~508).
  \item Footnote style: \verb|\footcite{reisner2013}| prints a full note with URL and access date.
\end{itemize}

\section{Good Practice}
\begin{itemize}
  \item Be consistent: choose one citation mode and stick to it.
  \item Prefer official/primary sources; include DOIs where available (\texttt{doi=\{...\}}).
  \item For online sources, always include \texttt{url} and \texttt{urldate}.
  \item Keep measurement units consistent across the thesis; reflect them in tables/figures, not the \texttt{.bib}.
\end{itemize}