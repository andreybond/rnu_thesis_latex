% \chapter{Methodology, Data, and Results}

% \section{Data}
% Describe datasets, measurement units, and preprocessing. Ensure units remain consistent throughout the thesis.

% \section{Methods}
% Present formulas as part of sentences, with numbering per chapter:
% \begin{equation}
%   \hat{y} = \beta_0 + \beta_1 x_1 + \cdots + \beta_k x_k + \varepsilon,
% \end{equation}
% where $\hat{y}$ — predicted value; $x_i$ — predictors; $\beta_i$ — coefficients; $\varepsilon$ — error term.

% \section{Results}
% Tables have captions above; figures below. Always analyze and reference them in text.

% \begin{table}[ht]
% \centering
% \caption{Sample Results by Group (units)}
% \label{tab:sample}
% \begin{tabular}{lrr}
% \hline
% Group & Mean & Std \\
% \hline
% A & 10.2 & 1.4 \\
% B &  9.6 & 1.2 \\
% \hline
% \end{tabular}
% \end{table}

% As shown in Table~\ref{tab:sample}, ...

% \begin{figure}[ht]
% \centering
% % Replace with your own figure includegraphics if needed
% \rule{0.8\linewidth}{0.35\linewidth}
% \caption{Illustrative Placeholder for a Figure.}
% \label{fig:placeholder}
% \end{figure}

% Discuss Figure~\ref{fig:placeholder} here.

\chapter{Formatting the Paper}

\section{Page and Text Layout}

The thesis shall be typeset on standard A4 pages (210\,mm $\times$ 297\,mm) in portrait orientation. Print on \emph{both} sides, except for the title page, abstracts, and the table of contents.

\subsection*{Page margins}
Use mirrored margins for two-sided printing:
\begin{itemize}
  \item \textbf{Left (inner):} 3\,cm
  \item \textbf{Right (outer):} 2\,cm
  \item \textbf{Top:} 3\,cm
  \item \textbf{Bottom:} 2\,cm
\end{itemize}

Page numbers are Arabic numerals placed in the \emph{outer} upper corner: on right-hand (odd) pages in the upper right, on left-hand (even) pages in the upper left. The first page is the title page, the second is the Table of Contents—both unnumbered (but counted). For graduate papers: Abstract on page 2 and Table of Contents on page 3 (both unnumbered). Numbering starts from the first page of the Introduction. All pages (including unnumbered) are counted in the total.

\emph{Mirror margins} should be enabled so inner/outer swap on verso/recto pages. Figure~\ref{fig:pagemargins} illustrates the page margin settings when printing on both sides.

\begin{figure}[ht]
  \centering
  % Replace with an actual screenshot if desired:
  \rule{0.8\linewidth}{0.35\linewidth}
  \caption{Page margin settings when printing on both sides.}
  \label{fig:pagemargins}
\end{figure}

\subsection*{Typography and spacing}
Main text is 12\,pt in Times New Roman or Calibri with 1.5 line spacing and black text. Paragraphs are fully justified. 

\subsection*{Headlines}
\begin{itemize}
  \item Headlines of parts, sections, and subsections are in \textbf{Bold}. Use small letters with the first letter of each meaningful word capitalized. Do \emph{not} hyphenate or underline; do \emph{not} add a full stop.
  \item Avoid long headlines (recommendation: at most two lines); if they wrap, the line spacing within the headline is single.
  \item If two headlines follow consecutively (e.g., a section immediately followed by a subsection), set the spacing between them to 12\,pt.
\end{itemize}

\subsection*{Vertical spacing around headings (narrative guidance)}
Within a part, the first paragraph starts 12\,pt below the part headline. In sections and subsections, leave 24\,pt above the headline relative to the preceding text and begin the following text 12\,pt below the headline.

\subsection*{Language and structure}
Write in grammatically and stylistically correct academic English:
\begin{itemize}
  \item Exposition should be precise, clear, logical, reasoned, and specific.
  \item Introduce each new idea in a new paragraph starting with an indent.
  \item Start each \emph{part} on a new page; sections and subsections continue after the previous text.
  \item Each part has an ordinal number (Arabic numerals) and a title; section numbers relate to part numbers; subsection numbers relate to section numbers.
  \item Titles in the text must match the Table of Contents exactly.
  \item The titles of parts—Introduction, Conclusions and Proposals, List of References—are preceded by numbers.
  \item Always provide references when using quotations, numbers, figures, tables, formulas, etc.
\end{itemize}

\textbf{Important:} Parts, sections, and subsections may neither \emph{start} nor \emph{finish} with tables, figures, formulas, or a list of sub-clauses. Begin with your own introductory text end with your own analysis.

\section{Tables}

Use tables to present numerical information or to clarify structure, without overloading the text.

\subsection*{Numbering and titles}
Each table has an ordinal number within the part. Use Arabic numerals and place the number \emph{above} the table, right-aligned. The word ``Table'' is capitalized. In ``Table~2.1'', the first digit is the part number and the second is the sequence within that part. Provide a concise title \emph{below} the number (i.e., still above the table), written in lowercase with a capitalized first letter and no full stop. If taken from another source, add the reference in brackets after the title.

\subsection*{Structure and units}
Name all columns and rows clearly. Titles should answer: \emph{What? Where? When?} Provide units where relevant.
\begin{itemize}
  \item Column/row titles are capitalized; subtitles are lowercase if forming a sentence with the title, otherwise capitalized.
  \item If a table continues onto another page, number columns and repeat the column numbers and headers on the new page, preceded by ``Table~2.1 Continued''.
\end{itemize}

\subsection*{Referencing and placement}
Refer to tables by number in the text or by abbreviation in brackets: e.g., (Tab.~2.1). Place tables immediately after their first mention and center them. Large tables (more than two-thirds of a page) should go to the Appendices; the main text must include the reference and analysis. In exceptional cases, larger tables may be used with prior approval.

\subsection*{Numbers and units in tables}
Avoid unwieldy large numbers; prefer scaled units (e.g., use $1\,000$~thousand pcs or $1$~m pcs instead of $1\,000\,000$~pcs). Use at most two decimal places unless required. Use standardized SI units. If all data share the same unit, show it in the table headline; if mixed, provide separate columns or indicate in column names.

\subsection*{Line spacing and font size in tables}
Use single line spacing inside tables. For large tables, you may reduce the font size to 10\,pt.

\paragraph*{Examples (in text)}
``The figures of the Erasmus{+} Mobility in 2015--2016 are presented in Table~2.1.'' \\
``In the last five years, an increase in the number of participants in Erasmus{+} Mobility program has been observed at RTU (see Tab.~2.1).''

\section{Figures}

All illustrations (schemes, diagrams, charts, photos, logos, etc.) are \emph{Figures}. Use figures to support the discussion; avoid excessive illustration without analysis.

\subsection*{Numbering, names, and captions}
Number figures with Arabic numerals within the part. Provide the figure number and name \emph{below} the figure on the same line. Write the figure name in lowercase with a capitalized first letter and add a full stop at the end. ``Figure~2.1'' means the first figure in Part~2.

\subsection*{Referencing and placement}
Place figures immediately after first mention, centered. Refer by name or with the abbreviation in brackets: (Fig.~2.1).

\paragraph*{Examples (in text)}
``The structure of the state budget of the Republic of Latvia is presented in Figure~2.1.'' \\
``There are five segments of the foreign exchange working as one united market (see Fig.~2.2).'' \\
``When choosing a HEI, foreign students acknowledge academic quality, international environment, and student focus as most important (see Fig.~2.3).''

\subsection*{Multiple sources}
If a figure (or a table) is based on more than one source, place the references beneath the item or in a footnote. For example:
\begin{quote}
\textit{Based on:}\\
1.~Eurostat database (2016) [online]. \textit{Gross domestic product at market prices}. [accessed 10 March 2016]. Available at: \texttt{<URL>}\\
2.~Central Statistical Bureau (2016) [online]. \textit{Long-term net migration}. [accessed 15 April 2016]. Available at: \texttt{<URL>}
\end{quote}

\subsection*{Size, orientation, and language}
Figures larger than two-thirds of a page should go to the Appendices with references and analysis in the main text. In exceptional cases and with adviser approval, large figures may be placed in landscape on a full page. Always analyze the information; do not place two figures in a row without intervening analysis.

Use single line spacing in figure text. Text within figures should be in English; exceptions for photos may be allowed by the adviser. For large figures, font size 10\,pt is recommended. For figures taken from literature or other sources, provide the original source in brackets after the name.

\section{Formulas}

Number formulas with Arabic numerals within the part. Place the number in parentheses, right-aligned on the formula’s baseline:
\begin{equation}
  \widehat{y} = \beta_{0} + \beta_{1}x_{1} + \cdots + \beta_{k}x_{k} + \varepsilon. \label{eq:reg}
\end{equation}
A formula is part of a sentence; follow it with the required punctuation (comma, semicolon, full stop, etc.) placed \emph{within} the formula environment.

Immediately after the formula and its number, provide an explanation starting with the word ``where'' on the left (no colon), decoding symbols and giving units:
\begin{quote}
\textbf{where} $\widehat{y}$—predicted value (unit); $x_i$—predictors (units); $\beta_i$—coefficients; $\varepsilon$—error term.
\end{quote}
Keep the unit of measurement for each value consistent across the entire paper.

Refer to formulas by number in the text, e.g., ``see~(\ref{eq:reg}).'' Provide references to the sources where formulas are published if they are not your own derivation.

\paragraph*{Example (textual reference)}
``The role of dummy variables in the regression can be considered in Formula~(2.3) (Gujarati, 2011, p.~47).''

\section{References}

To support academic integrity and honesty, adhere to the rules for citation and referencing. Literal copying without reference is a violation of copyright. Do not compile a paper by stitching fragments from sources without comparison, analysis, or your own assessment. The volume of quoted/copied text must not exceed the volume authored by you.

Use \emph{one} of the two referencing options consistently across the thesis.

\subsection*{Option 1: In-text (author--year, page)}
Provide the author’s surname in brackets, then year and page:
\begin{quote}
(Šatrevičs, Gaile-Sarkane, 2015, p.~94) \quad (Porter, 2008, p.~34; Lapiņa \emph{et~al.}, 2016, p.~269)
\end{quote}
If the source has two authors, name both; for more than two, name the first followed by \emph{et~al.}. Separate multiple sources with a semicolon.

\subsection*{Option 2: Footnotes}
Provide author(s), title, publisher, place and year, page number(s) in a footnote. Number either per page (starting at 1) or continuously through the thesis.
\begin{quote}
\textsuperscript{1} Kotler, P.\,T. \emph{et~al.} (2017). \textit{Marketing for Hospitality and Tourism}. 7th ed. London: Pearson Education Limited, p.~508.\\
\textsuperscript{2} Witte, A. (2014). \textit{Blending Spaces: Mediating and Assessing Intercultural Competence in the L2 Classroom}. Berlin: Walter de Gruyter GmbH \& Co KG, p.~383.
\end{quote}
When citing the same work repeatedly on one page, give the full description once, then use ``Ibid'' with the page number:
\begin{quote}
\textsuperscript{1} Kotler, P.\,T. \emph{et~al.} (2017). \textit{Marketing for Hospitality and Tourism}. 7th ed. London: Pearson Education Limited, p.~508.\\
\textsuperscript{2} \textit{Ibid.}, p.~55.
\end{quote}

\subsection*{Quotations and paraphrases}
Put exact quotations in quotation marks and place the reference either after the author’s name in-text or at the end of the quote (behind the closing marks and before the full stop). Use single/double inverted commas to highlight words/phrases being discussed. Rendered ideas (paraphrases) are not enclosed in quotation marks; they must be objective and accurate. If a reference applies to multiple sentences or a paragraph, place it after the last sentence.

\subsection*{Data and unpublished materials}
Numerical material used from company statistics, unpublished sources, etc. (data, calculations, charts) must be referenced with a clear mention of the data source(s), following the bibliography instructions. Where your own calculations are presented, include references to the original data sources used in those calculations.

