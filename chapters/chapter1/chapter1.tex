% \chapter{Theoretical Framework and Literature Review}

% Start sections with your own text; avoid starting/ending a section with a table/figure/formula or a raw list.

% \section{Key Concepts}
% Define key terms and concepts precisely.

% \section{Related Work}
% Summarize prior research critically; avoid excessive quotation without analysis \parencite{sedlacek2011,porter2008}.

% \section{Summary of Theoretical Insights}
% Conclude the chapter with bullet-point conclusions that will guide the empirical chapter.


\chapter{Methodological Guidelines}

The curricula of various study programs and the syllabi of individual study courses envision that alongside other academic activities and performance tests students develop and present various study papers—reports, projects and other individual or group papers. If internship forms a part of the program curriculum, students have to draw internship reports and publicly present them upon completion. The abovementioned papers are included in the study program curricula to ensure students acquire competences and skills in theoretical knowledge systematization, information analysis and practical application required to develop a graduate paper and take the \emph{viva} (viva voce)—publicly defend the results of their research.

\section{Paper Design}

The guidelines concerning thematic scope and content of a paper are determined by the head of each study program and the responsible member of academic staff. Paper development process includes several stages:
\begin{itemize}
  \item selection of the theme/research problem;
  \item paper structure design and development process planning;
  \item summarizing information and data;
  \item data analysis and systematization within the relevant theoretical framework;
  \item drawing and formatting the paper.
\end{itemize}

Survey and gathering of information is an important stage of paper design. As the required data are being summarized, partial data processing and systematization takes place, the data are also checked for relevance and sufficiency. A student has to comprehensively and thoroughly research theoretical and practical aspects of the problem to be solved, to evaluate critically the available bibliographic sources and empirical data.

Study and graduate papers have several compulsory constituent parts:
\begin{itemize}
  \item title page (See Appendixes 1–3);
  \item abstracts (in Latvian and English; only for graduate papers);
  \item table of contents (See Appendix 4);
  \item introduction;
  \item framework of the research (the body), which is divided into parts, sections and subsections in accordance with the requirements concerning paper content;
  \item conclusions and proposals;
  \item list of references (See Appendix 5);
  \item appendixes (if required);
  \item acknowledgement (if necessary);
  \item declaration of academic integrity of the graduate paper author (only for graduate papers);
  \item work performance and assessment sheet (only for graduate papers).
\end{itemize}

In accordance with paper structure and design, the student develops the body of the paper drawing relevant conclusions at the end of each part and/or section. Conclusions and proposals made at the end of the paper should be substantiated, proven and reasoned. Finally, the student shall format the paper in conformity with the \emph{Formatting and Style Guidelines for Study and Graduate Papers} and shall submit it for evaluation in accordance with the set procedure.

\section{Parts of the Paper}

The structure of the paper comprises the following parts: introduction, the body, conclusions and proposals.

\subsection{Introduction}

Introduction is an important part of the paper, which includes but is not limited to the following:
\begin{itemize}
  \item substantiation of the topicality of the theme;
  \item hypothesis or research questions (if necessary);
  \item aims/goals of the paper;
  \item formulation of the tasks/objectives necessary to achieve the aim of the paper;
  \item substantiation of the limitations to paper aims and tasks;
  \item list of research and planning methods and techniques employed;
  \item research period;
  \item list of references and literature sources;
  \item other information specified by the responsible member of academic staff in accordance with the study paper requirements.
\end{itemize}

The author of the paper sets a list of research questions or formulates the hypothesis in the form of a statement that has to be either proved or rejected. Based on the topicality of the selected theme, the author sets and formulates the aim of the paper. The formulation shall be brief and specific; it is recommended to specify measurable expected results. In accordance with the aim of the paper, the author sets the tasks that should be performed to achieve the aim. The set tasks determine the framework of the paper, as each task is addressed in the respective section of the paper. Limitations of the theme should be considered when it is not possible to comprehensively research all issues pertaining to the problem within the scope of the given paper.

\paragraph{Research methods.}
Research methods are tools used to complete the research tasks. The more accurately the research methods are selected, the more successfully the aim of the paper is going to be achieved. The range of available research methods is wide; therefore, the student should clearly specify in the introduction which methods will be used in the present research. The research methods used in the paper may include:
\begin{itemize}
  \item general research methods (analysis and synthesis, abstraction, inductive and deductive method, monographic or descriptive method, etc.);
  \item mathematical and statistical methods (estimation, comparison, correlation, grouping, etc. of mean and relative values, indexes, etc.);
  \item linguistic research methods (contrastive, corpus, text and discourse analysis, semantic and pragmatic methods, terminology research, intersemiotic research, etc.);
  \item empirical research methods (survey, observation, interview, case study, expert method, etc.);
  \item research methods topical for the specific industry or professional field.
\end{itemize}

The obtained information is reflected in tables and figures; the included information shall be clarified and commented upon and evaluation of regularities shall be provided in all cases.

\paragraph{Sources of the research.}
Sources of the research may include:
\begin{itemize}
  \item fundamental scientific research and monographs;
  \item articles from scientific journals and conference proceedings;
  \item publications on results of basic and applied research;
  \item general and special literature, periodicals;
  \item text corpora, databases (dictionaries, glossaries, encyclopaedias);
  \item laws, regulations of the Cabinet of Ministers, EC regulations and other regulatory documents, standards;
  \item statistical data;
  \item unpublished materials of companies, institutions and other organizations;
  \item different internet resources.
\end{itemize}

The authors can also refer to the results of polling, interviews, etc. and special surveys, as well as experiments performed under supervision or with participation of the author.

\subsection{The Body of the Paper}

The body of the presented research material is divided into parts. It is expedient to divide volume-wise large parts into sections, which can be further subdivided into subsections. The number of parts, sections and subsections depends on the paper volume and content. When the paper is organized into sections and subsections, their volume may not be less than two pages. A part may not have just one section and a section may not have just one subsection.

The task of the body of the paper is to provide description of the investigated problem, data analysis, results, conclusions and proposals of the author’s research in a systematized way. All parts should include illustrative materials and/or calculations: analytical tables, figures, formulas, etc.

\subsection{Conclusions and Proposals}

The final part of the paper comprises conclusions and proposals (if applicable). They are formulated as theses and are numbered separately—first conclusions and then proposals. Exceptions are possible, if determined otherwise by the responsible member of academic staff.

Conclusions reflect the most important ideas, which are based on the results of the research performed in the paper and provide solutions to the aims and tasks formulated in the introduction. Conclusions explain theoretical and practical significance of the performed research and the author’s personal achievement in solving the tasks. Conclusions proceed from the paper and may not be substantiated with the data and facts, which are not mentioned in the paper. Conclusions may not contain quotes from the works of other authors, but only personal views, ideas and opinions of the student.

Proposals proceed from the research conducted in the paper and are based on the conclusions made; they should be specific, substantiated and, whenever possible, addressed. Proposals summarize recommendations, offer improvements to be introduced or positive experience to be used, if such is mentioned and substantiated in the paper.