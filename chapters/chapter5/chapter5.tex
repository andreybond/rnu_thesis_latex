\chapter{Management of the List of References using Mendeley Tool}

Mendeley is a free tool, which allows managing one’s library by providing the following
functions:
\begin{itemize}
  \item to add PDF files to one’s library;
  \item to import/export files from other bibliography management tools, for example,
  EndNote, BibTex, and RIS;
  \item to find documents online using document identifier (for example, DOI) or document name and add them to one’s library;
  \item to import references with one click from the webpage opened in the browser, for example, EBSCO or SpringerLink, etc.;
  \item to manage one’s documents and bibliographic references;
  \item to add bibliographic references according to the chosen citation standard in MS Word
  and Open Ofice documents and automatically create a list of bibliographic references;
  \item to share documents in the Mendeley social network.
\end{itemize}

\section{User Registration}

To be able to use opportunities provided by Mendeley, it is necessary to register on Mendeley webpage \texttt{www.mendeley.com}. When registering, it is recommended to use your RTU e-mail in order to be able to use all advantages provided by the university.

Mendeley tool has several user interfaces:
\begin{itemize}
  \item \textbf{Mendey Web} – social network for cooperation and communication with other researchers, shared use of documents, creation of own research group library;
  \item \textbf{Mendeley Desktop} application, which should be installed on the computer. Mendeley Desktop application provides opportunities to manage one’s library and references. Synchronization between Mendeley Web and the local application is provided;
  \item \textbf{Mendeley Mobile} – application for mobile devices with limited functionality in comparison with Mendeley Desktop, it is envisioned for document reading and search, rather than for their processing.
\end{itemize}

\section{Library Creation}

Mendeley tool provides opportunity to create a library of sources, with or without full texts. This tool can be used both to collect bibliographic information on the read papers and to form a list of unread papers. Using the tool, one can group sources in a simple way, update source descriptions, automatically import descriptions of the sources from PDF files and internet resources.

PDF is the most common source file format, that is the reason why Mendeley tool provides ample opportunities to work with this format.

PDF files can be added to your library by clicking on \emph{Add Files} or by using the Drag \& Drop function when putting a file in the \emph{All Documents} list.

\begin{figure}[ht]
  \centering
  \rule{0.8\linewidth}{0.35\linewidth}
  \caption{Adding a PDF file.}
  \label{fig:mend-add-pdf}
\end{figure}

Mendeley will automatically try to determine bibliographic data of the attached PDF document – name, publication year, pages.

\begin{figure}[ht]
  \centering
  \rule{0.8\linewidth}{0.35\linewidth}
  \caption{Automatic determining of the bibliographic data.}
  \label{fig:mend-auto-meta}
\end{figure}

If all bibliographic data are full and correct, you can click \emph{Details are Correct}. Alternatively, you can click \emph{Search}, and then Mendeley will try to find information about this document online and supplement it. The user can also supplement the missing information himself/herself, if Mendeley fails to provide full bibliographic data.

It is recommended to complete the DOI field, when a document is ascribed a DOI. When using DOI, Mendeley can obtain the missing bibliographic data. It is recommended to provide as much bibliographic data as possible, as they will be of use later, when references and a list of references are drawn up automatically.

Mendeley provides opportunity to import/export documents from other bibliographic reference tools, for example, EndNote, BibTex and RIS.

Another opportunity to add documents to your library is to enter the document ID, for example, DOI, and make Mendeley find documents online. Choose \emph{File / Add Entry Manually} and enter DOI. Click on a button, which resembles a magnifying glass, and Mendeley will find the document. Click \emph{Save} and the document will be saved in the Mendeley library.

\begin{figure}[ht]
  \centering
  \rule{0.8\linewidth}{0.35\linewidth}
  \caption{Adding a source using DOI.}
  \label{fig:mend-add-doi}
\end{figure}

\emph{File / Add Entry Manually} can be used to enter descriptions of other sources. When using the tool, the type of source should be selected and the required fields completed.

\begin{figure}[ht]
  \centering
  \rule{0.8\linewidth}{0.35\linewidth}
  \caption{Adding different sources using \emph{Add Entry Manually}.}
  \label{fig:mend-add-manual}
\end{figure}

It is possible to add documents to your library also from a websites opened in a browser. For example, when an article, which you want to use in your paper, is opened in the IEEE database, you can import it into your Mendeley library by one click. This function is called \emph{One-Click Web Importer}.

First, the tool should be activated from the \emph{Tools / Install Web Importer} menu.

\begin{figure}[ht]
  \centering
  \rule{0.8\linewidth}{0.35\linewidth}
  \caption{Activation of the Web Importer tool.}
  \label{fig:mend-web-importer}
\end{figure}

The manual will provide explanations how Web Importer operates in different browsers – it should be added to the pinned browser bookmarks and should be called out, when import from an open webpage is needed.

For example, a publication to be quoted in your paper is open in the IEEE database. There is \emph{Save to Mendeley} button in the bookmark browser, which should be clicked.

Pressing \emph{Save to Mendeley} button, a window opens, where bibliographic data of the publication are provided. If everything is correct, press \emph{Save} and the publication will be automatically added to the Mendeley library, and you will be able to add references to it in your paper. Imported publications can be corrected or opened in the Mendeley application at once.

Mendeley provides opportunity to supplement one’s Mendeley library automatically also in case a document the user wants to add to his/her library is saved in a certain folder.

\begin{figure}[ht]
  \centering
  \rule{0.8\linewidth}{0.35\linewidth}
  \caption{Adding sources from folders.}
  \label{fig:mend-watched-folders}
\end{figure}

For example, when the user searches for literature sources according to key words in a database subscribed by RTU and saves the found documents in PDF format in his/her computer folder to read them later, Mendeley can initiate the function when all PDF files stored in this folder are automatically added to the Mendeley library, so that later it is possible to provide references to them and quote. This function is called \emph{Watched Folders}. It can be reached from \emph{Tools / Options / Watched Folders}, and the folder should be marked to add the saved PDF documents to the Mendeley library.

Mendeley provides opportunity to synchronize Mendeley Desktop with Mendeley Web, for example, when the user wants to access his/her library online from any place having authorized on Mendeley Web site. To be able to use the Sync function, the synchronization function should be activated. This can be done from \emph{All Documents / Edit Settings}. Mark the \emph{Synchronize attached files} menu checkbox. It is possible to choose all documents of the library to be synchronized or only documents from a specific folder. One has to monitor one’s occupied space volume (in GB) not to exceed the volume provided free of charge.

\begin{figure}[ht]
  \centering
  \rule{0.8\linewidth}{0.35\linewidth}
  \caption{Library synchronization.}
  \label{fig:mend-sync}
\end{figure}

Having pressed the \emph{Sync} button, synchronization between the user’s Mendeley Desktop and the Mendeley Web environments takes place, and, following the authorization in the Mendeley Web environment, the user can get access to the PDF documents in his/her library, as well as share them with others.

For additional mobility, it is possible to use the Mendeley mobile application, which is available on iOS and Android platforms. It has limited functionality and is envisaged for editing descriptions of existing sources and adding of new ones.

\section{Document Management}

The Mendeley tool provides extensive opportunities for management of documents. Documents can be marked as read or unread. When documents are added to the Mendeley library, they are marked as unread; their user can follow what new entries are attached and what should be read. When a document is open in Mendeley PDF Viewer, the document is marked as read.

\begin{figure}[ht]
  \centering
  \rule{0.8\linewidth}{0.35\linewidth}
  \caption{Status of a document in the Mendeley tool.}
  \label{fig:mend-status}
\end{figure}

A document can be open in Mendeley PDF Viewer by clicking on it twice. A tool bar is provided in Mendeley PDF Viewer, which allows putting marks in the text, highlighting the text, etc. A search opportunity entering key words is provided in the PDF document.

Mendeley provides opportunity to organize the library documents in different ways by using the \emph{File Organizer} tool – \emph{Tools / Options / File Organizer}. It is possible to copy all documents of the library from a specific folder and create a structure of folders by years, authors, journals, as well as rename each file added to the library according to the selected principle, for example, include the author, year and name in the file’s name.

\begin{figure}[ht]
  \centering
  \rule{0.8\linewidth}{0.35\linewidth}
  \caption{Grouping, arranging and renaming the sources.}
  \label{fig:mend-file-organizer}
\end{figure}

Mendeley provides opportunity to clean the library from duplicates by choosing \emph{Tools / Check for Duplicates}. It is possible to unite duplicates by pressing \emph{Confirm Merge}, when the documents have no contradictions in the bibliographic data.

\section{Citation of Documents and Generation of the List of References}

When a user has created own library, Mendeley provides opportunity to add references to Mendeley library documents in the documents created by the user and quote documents in conformity with international standards in MS Word and Open Ofice text editors. The list of references with the used citations is generated automatically, based on the chosen style, significantly reducing the administrative load of a researcher/student in developing their research or graduate paper.

To use this function, the MS Word or Open Ofice Mendeley plug-in should be activated. This can be done from the \emph{Tools / Install MS Word Plug In} menu.

\begin{figure}[ht]
  \centering
  \rule{0.8\linewidth}{0.35\linewidth}
  \caption{Activating MS Word plug-in.}
  \label{fig:mend-word-plugin}
\end{figure}

When the plug-in is activated, a new group \emph{Mendeley Cite-O-Matic} will appear in the MS Word \emph{References} tab. First in this section a citation style should be selected in the \emph{Style} field, which is used for formatting citations and the list of references. If the required style is not seen at once, click \emph{More Styles} and find the required.

\begin{figure}[ht]
  \centering
  \rule{0.8\linewidth}{0.35\linewidth}
  \caption{Mendeley Cite-O-Matic group in the MS Word Reference tab.}
  \label{fig:mend-cite-o-matic}
\end{figure}

American Psychological Association 6th Edition (APA) is the style that most closely meets the requirements of these Guidelines. By clicking \emph{Insert Citation}, the Mendeley window opens, where it is possible to find a document to be quoted. Citations can be searched by using any key word: author, name or its part, year, journal, etc.

\begin{figure}[ht]
  \centering
  \rule{0.8\linewidth}{0.35\linewidth}
  \caption{Reference search dialog box.}
  \label{fig:mend-ref-search}
\end{figure}

When selecting the right document among the found ones, press \emph{OK} button and the reference will appear in the text in conformity with the selected standard.

After all citations are placed, the List of References can be created. The list will include only the sources quoted in the paper. The list will be drawn up in accordance with the selected referencing style. To insert the list, click \emph{Insert Bibliography}. The inserted field will be automatically updated including new citations added to the document.
