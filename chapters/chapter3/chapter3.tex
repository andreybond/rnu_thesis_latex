\chapter{Formatting the List of References}

Compiling the List of References, bibliographic conventions for referencing should be observed. Sources are presented in the language they are written, and they are grouped in the alphabetical order according to the surname of the author or the first letter of the title of the work (material). The list is started with the sources in accordance with the Latin alphabet order followed by the sources in the Cyrillic alphabet order. Bibliography and other sources written in Russian are grouped separately, as the sequence of letters in Latin and Cyrillic alphabets differ.

The list should include all sources of information used in the paper in the alphabetical order (see Appendix~5). Sources in the references are provided in the original language, either the Latin or Cyrillic alphabet. When a reference is provided to a source published in other languages (except for English or Russian), for example, Arabic or Chinese, the reference should be transliterated in the letters of the Latin alphabet.

Referring to a text where pages are numbered, the number of the specific page, where the information has been taken, shall be provided in the reference. The total number of pages shall be provided in the List of References. Abbreviation of the word page, similarly to the source, shall be used in the original language of the source.

\noindent\textbf{Example:} \emph{40 lpp.} or \emph{40 стр.}, or \emph{40 p.}

The List of References is drawn using Single Line Spacing between the rows of one source description; the spacing between different sources is one Spacing of 6\,pt.

\section{Books}

Data about books (monographs, textbooks, manuals, summaries of doctoral theses, etc.) should be provided as follows: \emph{The author's (authors') surname, initials. (Year of publication in brackets). Title of the book (in Italic). Issue (if required). Place of publication: Publisher. Volume or total number of pages. Full stop at the end.} For an e-book, a reference to the website is provided as well: \emph{Available at: \textless URL\textgreater}.

\paragraph{Reference to a book by one author (in the body of the paper)} The process of achieving the national and regional energy efficiency targets involves several systems, therefore, use of the process approach allows gear them to a closer cooperation \parencite[26]{actina2015}. The e-market is characterized by industrial and technological convergence, disappearance of physical boundaries, changes in consumer and retailer demands, smart and creative industry competition, etc. \parencite[19]{sceulovs2015}. Today, science enjoys hiding behind an ivory wall \parencite[11]{sedlacek2011}.

\paragraph{Entry in the List of References (examples)}%
\begin{itemize}
  \item Actiņa, G. (2015). \emph{Development System of Energy Eficient Processes in Latvia. Summary of Doctoral Thesis.} Riga: RTU Press. 70~p.
  \item Sceulovs, D. (2015). \emph{Use of Electronic Environment in Entrepreneurship Development. Summary of Doctoral Thesis.} Riga: RTU Press. 47~p.
  \item Sedláček, T. (2011). \emph{Economics of Good and Evil: The Quest for Economic Meaning from Gilgamesh to Wall Street.} New York: Oxford University Press. 11~p.
\end{itemize}

\paragraph{Other examples}
\begin{itemize}
  \item Friedman, T.~L. (2006). \emph{The World is Flat.} London: Penguin Books. 600~p.
  \item Kaftan, M. (2005). \emph{Modern Business English in E-commerce.} Prague: LEDA Publishing House. 288~p.
  \item Porter, M. (2008). \emph{On Competition: Updated and Expanded Edition.} Boston, MA: Harvard Business School Pub. 544~p.
\end{itemize}

\paragraph{Book by 2--3 authors}
\emph{In-text examples:} Almost every company has ``tribal elders'' who possess unique expertise \parencite[183]{schmidt2014}. Cost accounting provides information for both management accounting and financial accounting \parencite[4]{horngren2015}. Language of science as an LSP is defined as the language ``used for particular and restricted types of communication'' \parencite[159]{richards1985}.

\emph{List entries:}
\begin{itemize}
  \item Horngren, C.~T., Srikant, M.~D., Madhav, V.~R. (2015). \emph{Cost Accounting: A Managerial Emphasis.} 15th ed. Boston: Pearson. 960~p.
  \item Schmidt, E., Rosenberg, J., Eagle, A. (2014). \emph{How Google Works.} London: John Murray (Publishers). 352~p.
  \item Richards, J., Platt, J., Weber, H. (1985). \emph{Longman Dictionary of Applied Linguistics.} Harlow: Longman. 159~p.
\end{itemize}

\emph{Other:}
\begin{itemize}
  \item Laudon, K.~C., Traver, C.~G. (2015). \emph{E-Commerce 2016: Business, Technology, Society.} Boston: Pearson. 905~p.
  \item Horngren, C.~T., Srikant, M.~D. (2003). \emph{Cost Accounting: A Managerial Emphasis.} 11th ed. New Jersey: Prentice Hall. 880~p.
  \item Гапоненко, А., Орлова, Т. (2008). \emph{Управление знаниями.} Москва: Эксмо. 550~стр.
\end{itemize}

\paragraph{Book by more than three authors}
\emph{In-text examples:} To compete in this aggressive environment, companies must shift their focus \parencite[1]{christensen2013}. Texts often exhibit the conventions of more than one genre \parencite[129]{osullivan1994}.

\emph{List entries:}
\begin{itemize}
  \item Christensen, C.~M. \emph{et al.} (2013). \emph{HBR's 10 Must Reads. On Strategic Marketing.} Boston: Harvard Business Review Press. 256~p.
  \item O'Sullivan, T. \emph{et al.} (1994). \emph{Key Concepts in Communication and Cultural Studies.} London: Routledge. 367~p.
\end{itemize}

If the reference is provided to a book that is a collection of articles or a work of a team of authors, it is included in the List of References in accordance with the first letter of its title; the notation ``edited by \dots'' is provided after the title.

\emph{Example:} \emph{The Routledge Handbook of Corpus Linguistics} (2010). Edited by: A.~O'Keeffe and M.~McCarthy. London; New York: Routledge. 682~p.

\section{Scientific Articles and Abstracts}

Data about articles in scientific journals should be provided as follows: \emph{The author's (authors') surname, initials. (Year of publication in brackets). Title of the article. Title of the publication (in Italic), issue, volume of the article (pages from--to).}

\paragraph{Reference in English (in-text examples)} Detailed analysis demonstrated that such result is due to influence of some individual countries \parencite[237]{semjonova2015}. The implementation of the transdisciplinary paradigm becomes even more complicated in the multilingual setting, as the ``communication strategies that are convincing in one language culture may be regarded as suspicious in another'' \parencite[209]{junge2011}.

\paragraph{List entries (examples)}
\begin{itemize}
  \item Semjonova, N. (2015). Assessment of the Government Debt Position Impact on the General Taxation Policy. \emph{Economic Science for Rural Development}, Vol.~37, pp.~232--240.
  \item Junge, S. (2011). Corporate Rhetoric in English and Japanese Business Reports. \emph{Multilingual Discourse Production: Diachronic and Synchronic Perspective}, edited by Kranich, S. \emph{et al.} Hamburg Studies on Multilingualism, Vol.~12. John Benjamins Publishing, pp.~207--232.
  \item Lapina, I., Roga, R., Müürsepp, P. (2016). Quality of Higher Education: International Students' Satisfaction and Learning Experience. \emph{International Journal of Quality and Service Sciences}, Vol.~8 Iss.~3, pp.~263--278.
\end{itemize}

\paragraph{Reference in a language other than the paper} If the author quotes sources written in languages other than the language of the paper, specify whether the quote was translated by the author. \emph{In-text example:}

\begin{quote}
В Латвии в расчёт финансового выравнивания самоуправлений включаются не все доходы муниципалитетов \dots{} (Шенфелде, Янсоне, 2012, 390, author's translation).
\end{quote}

\emph{List entry example:} Шенфелде, М., Янсоне, С. (2012). Финансовое выравнивание самоуправлений Латвии как фактор их развития. \emph{Проблемы развития внешнеэкономических связей и привлечения иностранных инвестиций: региональный аспект}, №~1, стр.~387--395.

\paragraph{Conference proceedings and thesis books}

\emph{Format:} The author's (authors') surname, initials. (Year). Title of the article in the source language. \emph{Title of the publication and the conference, issue (if required), date, month, year.} City: Publisher, pages from--to.

\emph{Examples:}
\begin{itemize}
  \item Šatrevičs, V., Gaile-Sarkane, E. (2015). Strategic Fit Relation Model as a Tool for Organization Development. \emph{Proceedings of the 19th World Multi-Conference on Systemics, Cybernetics and Informatics (WMSCI 2015), July 12--15, 2015}. Florida: International Institute of Informatics and Systemics, pp.~94--99.
  \item Иевинс, Я., Бартусаускис, Я., Мелько, А. (2013). Система охраны труда в Латвии, проблемы и их решение. \emph{Международный экологический конгресс ``Ecology and Life Protection of Industrial-Transport Complexes'', 18--22 сентябрь, 2013.} Тольятти: ТГУ, стр.~115--123.
\end{itemize}

\section{Internet Sources}

Materials obtained from the Internet are presented as bibliographic references to e-resources. It is required to provide further sites proceeding from the address. Often the year of publication and the publisher are not easily determined; provide them whenever possible.

\paragraph{Format A (with author):} The author's surname, initials. (Year). \emph{Title} (in source language, in italics). Publisher or site holder, or webpage name [date of viewing]. Available at: \textless URL\textgreater.

\paragraph{Format B (no named author):} \emph{Title} (Year) [online]. Publisher or site holder, or webpage name [date of viewing]. Available at: \textless URL\textgreater.

\emph{In-text examples:} The EU Customs Competency Framework \parencite{reisner2013}. Skills expected of translators are certain to change as well \parencite[1]{pym2012}.

\emph{List entries:}
\begin{itemize}
  \item Reisner, B. (2013). EU Competency Framework for the Customs profession [online]. PICARD Conference in St.~Petersburg, 18--20 September 2013 [accessed on 17 November 2013]. Available at: \verb|http://www.wcoomd.org/en/events/event-history/2013/wco-picard-conference-2013/~/media/09569F05EF4240D6A5CCE0327CEF03CC.ashx|
  \item Pym, A. (2012). Translation Skill Sets in a Machine Translation Age [online]. Universitat Rovira i Virgili, Tarragona: Spain (InterCultural Studies Group). [accessed on 10 March 2016]. Available at: \verb|http://usuaris.tinet.cat/apym/on-line/training/2012_competence_pym.pdf|
\end{itemize}

\paragraph{Legal sources without a named author (in-text example)} Section~2. Scope of Application of the Law (The Commercial Law, 2002):
\begin{enumerate}
  \item A merchant is a natural person (individual merchant) or a commercial company (partnership and capital company) registered with the Commercial Register.
  \item Commercial activity is an open economic activity, which is performed by merchants in their name for the purposes of gaining a profit. Commercial activity is one of the types of entrepreneurial activity.
  \item Economic activities are any systematic, independent activities for remuneration.
  \item In this Law it may be specified that particular types of economic activities may only be performed by a merchant. The status of a merchant may be granted by law also to other persons.
\end{enumerate}

\emph{List entry:} \emph{The Commercial Law} (2002) [online]. Law of the Republic of Latvia, adopted in Riga on 14 February 2002, Latvijas Vēstnesis website [accessed on 26 March 2017]. Available at: \verb|https://likumi.lv/ta/id/5490-komerclikums|

\paragraph{Multiple Internet sources (examples)} One of the tasks of the State Revenue Service is to identify criminal activities in the field of state taxes and other mandatory payments (State Revenue Service Business Strategy, 2014). A significant aspect \dots{} the Baltic Centre for Investigative Journalism has updated the contribution of banks (Jenberga, Puriņa, 2016).

\emph{List entries:}
\begin{itemize}
  \item State Revenue Service Business Strategy for 2014--2016 (2014) [online]. State Revenue Service website [accessed on 3 February 2016]. Available at: \verb|https://www.vid.gov.lv/default.aspx?tabid=4&id=684&hl=1|
  \item Jenberga, S., Puriņa, E. (2016). U.S. Pressures Latvia to Clean Up Its Non-Resident Banks [online]. Re:Baltica -- The Baltic Centre for Investigative Journalism [accessed 4 February 2016]. Available at: \verb|http://www.rebaltica.lv/en/investigations/dirty_money/a/1316/u_s%20pressures_latvia_to_clean_up_its_non-resident_banks.html|
\end{itemize}

\section{Dictionaries}

The reference to printed sources is provided as to a book. \emph{In-text example:} In both English and Latvian the lexical unit \emph{head} (Latv.: \emph{galva}) is one of the rare instances when metaphoric meaning transfer is used \parencite[582]{krauklis2003}.

\emph{List entries:}
\begin{itemize}
  \item Green, J. (2010). \emph{Green's Dictionary of Slang.} UK: Chambers.
  \item Krauklis, V. (2003). \emph{Dictionary of Terms in Civil Engineering // Celtniecības terminu vārdnīca.} Rīga: Telamons. 1327~p.
\end{itemize}

To promote readability the author may ascribe abbreviations to digital/online sources (e.g., dictionary, glossary, thesaurus or database) used in the paper and listed in the references.

A headword of the entry cited from any dictionary \dots{} should be given in quotation marks and followed by reference information in square brackets: number/letter of definition (if multiple) and title of the source (in italics).

\emph{In-text example:} The metaphoric terms that are based on the concept ``LEG'' [Def.~2a, \emph{MW}] include \dots{}

\emph{List entry:} MW -- Merriam--Webster Dictionary [online]. Available at \verb|https://www.merriam-webster.com/dictionary/leg| [accessed on 8 February 2017].

\section{Standards}

\subsection*{Standards approved by organizations or standardization bureaus and published on the Internet}

\emph{Format:} \emph{Title of the standard} (Year) [online]. Publisher or standard holder, pages (if provided) [date of viewing]. Available at: \textless URL\textgreater.

\emph{In-text examples:} This standard is based on a number of quality management principles (ISO~9001{:}2015, 2015). Currently, ISO/TC~37 works on establishing the basic principles and methods \dots{} (ISO/TC~37).

\emph{List entries:}
\begin{itemize}
  \item ISO~9001{:}2015 -- Quality Management [online]. International Organization for Standardization [accessed on 18 March 2016]. Available at: \verb|http://www.iso.org/iso/home/standards/management-standards/iso_9000.htm|
  \item ISO/IEC~2382--37 -- Terminology Standards [online]. The International Organization for Standardization [accessed on 3 April 2016]. Available at: \verb|http://www.iso.org/|
\end{itemize}

\subsection*{Standards issued and approved in accordance with laws or other regulatory enactments}

\emph{In-text example:} In accordance with the Latvian Accounting Standard No.~11 (2010), Clause~8, ``Costs of reserves include all acquisition, manufacturing and other expenses, which are incurred delivering reserves to their current location and condition.''

\emph{List entry:} \emph{Latvian Accounting Standard No.~11 ``Reserves''} (2010) [online]. Adopted with the resolution of the Accounting Council on 8 September 2010, Latvijas Vēstnesis website [accessed on 18 March 2016]. Available at: \verb|http://likumi.lv/doc.php?id=221422|

\subsection*{Standards published as a book}

The reference to a standard summarized and published as a book (author/team known) follows book guidelines. If issued as an e-book, follow e-book guidelines.

\emph{In-text example:} Access control and road charging systems may also use optical number plate recognition \parencite[86]{williams2008}.

\emph{List entry:} Williams, B. (2008). \emph{Intelligent Transport System Standards.} Boston; London: Artech House. 817~p.

\section{Company Materials, Interviews and Other Sources}

Developing study and graduate papers, ``unpublished materials'' are often used (internal company materials, interviews, theses, seminars, etc.). Such materials are presented similarly to books, journal publications, Internet sources, etc.

\subsection*{Company materials}

\emph{Format:} Title of the material (year). Company name and other specific information (if applicable). Volume (pages).

\emph{In-text example:} Eiklīda Ltd considerably expanded in 2015 \dots{} (Eiklīda, Ltd, Annual Report, 2016).

\emph{List entry:} \emph{Eiklīda, Ltd. Annual Report} (2016). Eiklīda Ltd accounting materials. 65~p.

\subsection*{Bachelor and Master papers publicly available at universities}

\emph{Format:} Author's surname, initials. (Year). \emph{Title} (Italic). Paper type. Place: University. Volume or pages.

\emph{In-text example:} Latvia is largely dependent on foreign suppliers of fuels \dots{} (de~Linde, 2016).

\emph{List entry:} de~Linde, S. (2016). \emph{Overcoming market barriers and promoting deep renovation of multifamily residential buildings in Latvia.} Master Thesis. Riga: Riga Technical University. 97~p.

\subsection*{Interview}

\emph{Format:} Surname, initials of the interviewee. (Year, date, month). \emph{Theme/name of the interview.} Surname, initials of the interviewer.

\emph{In-text example:} K.~Balodis (Interview, 8 January 2016) mentioned three main challenges \dots{}

\emph{List entry:} Balodis, K. (8 January 2016). \emph{Entrepreneurship Opportunities in Small Towns.} (Interviewed by I.~Odziņa).

\subsection*{Handouts and presentation materials of seminars and conferences}

\emph{Format:} Author's surname, initials. (Year, date). Title and type of the material. Name of the conference or seminar.

\emph{In-text example:} Digital world brings several opportunities for education \dots{} (Urva, 29 September 2016).

\emph{List entry:} Urva, E. (29 September 2016). \emph{Opportunities and challenges the increasingly digital world brings to educators.} ``Management Education for a Digital World'' 24th CEEMAN Annual Conference.
