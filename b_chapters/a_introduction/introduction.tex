\unchapter{Introduction}

The introduction outlines the relevance of the chosen topic, its aim, object and subject of study, formulated research problem, research tasks, and applied methods. It also may include the research hypothesis and information about the approbation (presentation, publication, or expert review) of the results.  

\textbf{Topicality of the problem:}  
Briefly describe why the selected topic is important and relevant. This may include current challenges in the industry, social or economic context, scientific novelty, or practical application.  

\textbf{Aim of the study:}  
Clearly define the main purpose of the work. The aim is usually expressed in one sentence and should not be too broad or too narrow. Example: “The aim of the study is to develop recommendations for improving [system/process/topic] in [context/field].”  

\textbf{Object of the study:}  
Indicate the broader field or phenomenon under investigation (e.g., “entrepreneurial activity,” “information systems,” “educational processes”).  

\textbf{Subject of the study:}  
Specify the narrower aspect within the object that will be analyzed (e.g., “employee motivation system,” “data warehouse design methodology,” “student performance evaluation methods”).  

\textbf{Research problem:}  
Formulate the main problem that the study addresses. This can be expressed as a contradiction, challenge, or knowledge gap (e.g., “high staff turnover,” “lack of effective data integration,” “insufficient methodological guidance”).  

\textbf{Tasks of the study:}  
List the specific steps needed to achieve the aim. Tasks are usually formulated as 4–6 items and should correspond directly to the structure of the thesis. For example:  
1. Review theoretical approaches and relevant literature.  
2. Analyze the chosen case, field, or system.  
3. Conduct empirical or practical research.  
4. Develop proposals, models, or improvements and evaluate their efficiency.  

\textbf{Hypothesis (if applicable):}  
Formulate a testable statement about the expected outcome. Example: “Implementing the proposed measures will improve service quality and reduce costs.”  

\textbf{Research methods:}  
Indicate the methods used in the work (e.g., literature analysis, surveys, experiments, case studies, statistical analysis, modelling, comparative analysis).  

\textbf{Approbation of the study (if any):}  
Mention where and how the results have been presented or evaluated (e.g., student conferences, publications, expert reviews).  
