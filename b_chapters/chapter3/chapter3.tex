\chapter{Empirical Study / Practical Research (Example Chapter~3)}
\label{chap:empirical}

% --- IMPORTANT NOTE FOR STUDENTS ---
% This is only a suggested structure. 
% You may adjust, merge, expand, or shorten sections depending on your topic,
% as long as the choices are approved by your supervisor.
% Some theses have a large empirical part, others a very small one. 
% Adapt to your own case.

This chapter usually presents the applied or empirical part of the thesis: an experiment, prototype, survey, simulation, or case study.  
However, the exact scope depends on your project. In some theses, Chapters~2 and~3 are merged into one; in others, this chapter may be split into two separate ones (e.g., “Implementation” and “Evaluation”).

\section{Design of the Study (Example Section)}
\label{sec:design}
Explain how your empirical/practical work was organized.  
Typical points to include (adapt as needed):

\begin{itemize}[leftmargin=1.2cm]
  \item \textbf{Objectives.} What you wanted to achieve or test.  
  \item \textbf{Setup.} Hardware, software, datasets, or infrastructure used.  
  \item \textbf{Participants or subjects (if any).} For surveys, experiments, or usability tests.  
  \item \textbf{Procedures.} How the study/experiment was carried out, in enough detail to be reproducible.  
\end{itemize}

\paragraph{Note.} If your work is implementation-heavy, this section can describe the system architecture, algorithms, or workflows you built. If it is more research-focused, describe the methodology and design choices.

\section{Results of the Study (Example Section)}
\label{sec:results}
Present the main outcomes clearly and systematically.  
Examples:

\begin{itemize}[leftmargin=1.2cm]
  \item Tables or charts with performance benchmarks, error rates, or runtime.  
  \item Screenshots or diagrams of a prototype or interface.  
  \item Summaries of survey/interview data.  
\end{itemize}

% --- Example figure (replace or remove) ---
\begin{figure}[h]
  \centering
  \includegraphics[width=0.6\linewidth]{b_chapters/chapter1/assets/RNU_large_logo.png}
  \figsource{Replace with your own chart or prototype screenshot.}
  \caption{Example chart of study results (dummy figure).}
  \label{fig:study-results}
\end{figure}

\paragraph{Tip.} Do not over-interpret here — keep deep discussion for Chapter~\ref{chap:discussion}.  
The focus here is on presenting what you observed.

\section{Evaluation and Interpretation (Optional)}
\label{sec:evaluation}
In some topics it is useful to add a subsection where you briefly evaluate your results against expectations or benchmarks.  
Examples: compare against a baseline algorithm, test statistical significance, or highlight unexpected patterns.  
If this does not fit your project, you can skip or merge it into the next chapter.

\section{Sub-Conclusions (for Chapter~3)}
\label{sec:subconcl3}
End with a short summary (2–3 paragraphs).  
Example points:

\begin{itemize}[leftmargin=1.2cm]
  \item Key empirical findings (what was achieved, measured, or demonstrated).  
  \item Relation to the aim and hypothesis from the Introduction.  
  \item How these findings prepare for the next step — either a discussion chapter (Chapter~\ref{chap:discussion}) or the final conclusions (Chapter~\ref{chap:conclusions}).  
\end{itemize}

\paragraph{Reminder.} The conclusions here should be short and local to this chapter. The overall “big picture” conclusions must still be saved for the dedicated \textbf{Conclusions chapter}.
