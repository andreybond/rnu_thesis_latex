\chapter{Empirical Study / Practical Research (Example Chapter~3)}
\label{chap:empirical}

% --- Guidelines for students ---
% Chapter 3 is often the empirical or applied part of the thesis:
% - field survey, experiments, prototype, modelling, case study, etc.
% The size and contents depend on your topic.
% Remember: quality and clarity are more important than length.

\section{Design of the Study}
Describe how your empirical work was set up: objectives, participants/subjects, instruments/tools, sampling method, procedures.

\section{Results of the Study}
Present your results in a structured way (tables, figures, text). Explain what you observed. Do not interpret too much here (save in Chapter 4 “Discussion”).

% Example: small figure with dummy data
\begin{figure}[h]
  \centering
  \includegraphics[width=0.6\linewidth]{b_chapters/chapter1/assets/isma_logo.png}
  \figsource{Replace with your own chart or diagram.}
  \caption{Example chart of study results.}
  \label{fig:study-results}
\end{figure}

\section{Sub-Conclusions (for Chapter~3)}
Briefly summarize the main empirical findings (2–3 paragraphs). Link them back to the aim and hypothesis, but keep the deeper discussion for the next chapter.

% --- Note ---
% If your thesis only has 2 chapters, merge Chapters 2 and 3 into one.
% If you need more space (e.g., empirical + evaluation), you may split
% into 3 or even 4 chapters. Adjust flexibly.