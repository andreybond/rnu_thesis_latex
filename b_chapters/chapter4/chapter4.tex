\chapter{Discussion and Development (Example Chapter~4)}
\label{chap:discussion}

% --- IMPORTANT NOTE FOR STUDENTS ---
% This chapter is optional and highly flexible.
% Some theses combine discussion with results (Chapter~3).
% Others use it to present proposals, improvements, or an evaluation.
% Always adapt the content/structure to your project and check with your supervisor.

This chapter may serve several possible purposes depending on the nature of your work:  
(1) discuss your empirical findings in more depth,  
(2) compare results against theory or prior work,  
(3) propose improvements, models, or design solutions, or  
(4) evaluate feasibility or efficiency.  
If your project is small-scale or self-contained, you may skip this chapter and move directly to the Conclusions.

\section{Extended Discussion (Optional Example Section)}
\label{sec:discussion-extended}
Here you may reflect more deeply on your results than in Chapter~\ref{chap:empirical}.  
Possible angles include:

\begin{itemize}[leftmargin=1.2cm]
  \item Comparing results with other studies or benchmarks.  
  \item Highlighting unexpected patterns or anomalies.  
  \item Theoretical implications: how your findings support or contradict existing models.  
\end{itemize}

\paragraph{Tip.} If your work is primarily an implementation, you may instead use this section to evaluate design trade-offs, architectural choices, or lessons learned.

\section{Proposals / Improvements (Optional Example Section)}
\label{sec:proposals}
In many IT theses, the contribution is not only empirical but also practical.  
Here you can suggest:

\begin{itemize}[leftmargin=1.2cm]
  \item Proposed system modifications or design improvements.  
  \item Prototype features, extensions, or optimizations.  
  \item Conceptual models or frameworks based on your findings.  
\end{itemize}

Illustrate with diagrams, mockups, or tables if useful. Remember: these are suggestions, not required elements.

\section{Evaluation of Feasibility / Efficiency (Optional Example Section)}
\label{sec:evaluation}
Sometimes it makes sense to estimate how well your proposal would work in practice.  
Examples:

\begin{itemize}[leftmargin=1.2cm]
  \item Performance evaluation (e.g., runtime, memory use, scalability).  
  \item Cost/benefit trade-offs (time, resources, usability).  
  \item Risks and limitations of implementation.  
\end{itemize}

This section can be skipped if it does not apply to your project.

\section{Sub-Conclusions (for Chapter~4)}
\label{sec:subconcl4}
Close with a short synthesis that ties together the discussion and proposals.  
Example points:

\begin{itemize}[leftmargin=1.2cm]
  \item What your deeper reflection adds beyond the raw results.  
  \item Which improvements or models are most promising.  
  \item How these insights prepare the way for the \textbf{Conclusions chapter}.  
\end{itemize}

\paragraph{Reminder.} The “big picture” statements — confirmation of tasks, validation of the hypothesis, and achievement of the aim — belong in the \textbf{Conclusions chapter}, not here.
