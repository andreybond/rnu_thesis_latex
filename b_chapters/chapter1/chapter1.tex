\chapter{Literature Review and Theoretical Background}
\label{chap:litreview}

% --- Purpose (short instructional paragraph for students) ---
% This chapter introduces key concepts/definitions, surveys related work,
% and presents the theoretical framework that underpins the thesis.
% Keep it focused: define terms, compare sources, and end with a brief synthesis.

\section{Core Concepts and Definitions}
\label{sec:concepts}
Briefly introduce the core terms and definitions your thesis relies on. Use primary/authoritative sources and cite them properly, e.g., \cite{porter2008}. When you use a term for the first time, define it clearly and consistently.

Inline emphasis works as expected: \emph{italics} and \textbf{bold}. Use \verb|\enquote{...}| (from \texttt{csquotes}) for proper quotation marks: \enquote{quoted text}. URLs should be clickable with \verb|\url{...}|, e.g., \url{https://ctan.org}.

\paragraph{Tip (citations).}
With \texttt{biblatex} (IEEE style here), the basic numeric citation is \verb|\cite{...}|. Place citations after punctuation when they refer to a whole sentence.\footnote{Footnotes are fine for clarifications, not for core claims.}

\section{Related Work: Structure and Comparison}
\label{sec:related}
Organize related work by themes, methods, or chronology. Summarize each strand and compare differences/restrictions. You can use lists to keep comparisons readable:

\begin{itemize}[leftmargin=1.2cm]
  \item \textbf{Approach A} — strengths, limitations, and where it applies \cite{friedman2006}.
  \item \textbf{Approach B} — contrast with A; mention datasets, measures \cite{laudon2015}.
  \item \textbf{Approach C} — hybrid or recent trend; open issues \cite{schmidt2014}.
\end{itemize}

When you need to point the reader to figures, tables, algorithms, or listings, always use labels and \verb|\cref|:
see \cref{fig:example-figure,tab:example-table,alg:example, lst:bubble}.

% ---- Figure example (title under figure; centered by class) ----
\begin{figure}[h]
  \centering
  \includegraphics[width=0.7\linewidth]{b_chapters/chapter1/assets/isma_logo.png}
  \figsource{Author-created illustration (replace with your own figure and source).}
  \caption{Example figure: visualizing a conceptual taxonomy (replace).}
  \label{fig:example-figure}
\end{figure}

% ---- Table example (title above; ragged-right by class) ----
{\singlespacing
\begin{table}[h]
  \caption{Example comparison table (replace with your synthesis)}
  \label{tab:example-table}
  \centering
  \begin{tabular}{lrrr}
    \toprule
    Criterion            & Method A & Method B & Method C\\
    \midrule
    Accuracy (avg)       & 0.82     & 0.79     & 0.84\\
    Cost (relative)      & High     & Medium   & Low\\
    Data requirement     & Large    & Medium   & Small\\
    \bottomrule
  \end{tabular}

  \vspace{2mm}
  \emph{Note:} Replace numbers with your findings; report units and sources.
\end{table}
}

\section{Theoretical Framework}
\label{sec:theory}
State the theory or model that grounds your work. Formally define variables and relations when appropriate. Use numbered equations for important results:

\begin{align}
  \label{eq:objective}
  J(\theta) &= \mathbb{E}_{x \sim \mathcal{D}} \bigl[\, \ell(f_\theta(x),\, y)\, \bigr]
\end{align}

Reference equations with \verb|\cref|, e.g., the objective in \cref{eq:objective}. Keep derivations concise and push long proofs to an appendix if needed.

% ---- Algorithm (float + pseudocode) example ----
\begin{algorithm}[h]
\caption{Generic greedy selection (template — adapt to your context)}
\begin{algorithmic}[1]
  \State Initialize solution $S \gets \emptyset$
  \While{a feasible choice exists}
    \State choose the best feasible item $x$ according to criterion $C$
    \State $S \gets S \cup \{x\}$
  \EndWhile
  \State \Return $S$
\end{algorithmic}
\label{alg:example}
\end{algorithm}

\section{Illustrative Code Snippet (if applicable)}
\label{sec:listings}
For reproducibility, small, focused code excerpts can clarify ideas. Use the \texttt{listings} environment; keep captions short and descriptive.

\begin{lstlisting}[language=Python,caption={Example function signature for data transform},label={lst:bubble},float=htbp]
def transform(records: list[dict]) -> list[dict]:
    """Validate and normalize input records.
    Replace with your own code; keep examples minimal and clear.
    """
    out = []
    for r in records:
        if "id" not in r:  # simple guard
            continue
        r = {k.strip().lower(): v for k, v in r.items()}
        out.append(r)
    return out
\end{lstlisting}

Reference listings with \verb|\cref{lst:bubble}|. If you need more advanced syntax highlighting, consider \texttt{minted} (already disabled in this template to avoid external deps).

\section{Chapter Summary}
\label{sec:summary}
End with a short synthesis that links
(1) definitions and concepts (\cref{sec:concepts}),
(2) gaps and comparisons in related work (\cref{sec:related}),
and (3) the theoretical lens you will apply (\cref{sec:theory}).
This summary should directly motivate the next chapter’s objectives and methods.

\noindent\textbf{Conclusions (for Chapter~1).}
Write a brief paragraph that: (i) states the scope of the literature you covered, (ii) highlights key gaps/opportunities your work addresses, and (iii) justifies the chosen theoretical framework and its relevance to your research aim.
