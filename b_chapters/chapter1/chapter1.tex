\chapter{Theoretical approaches to the study of employee motivation}
\section{Concept of employee motivation}
Provide definitions and literature review; end with brief sub-conclusions.

\section{Basic theories of employee motivation}
Compare classic and contemporary theories; outline implications.

\section{System approach to employee motivation}
\subsection{Principles of employee motivation}
\subsection{Types of employee motivation}

\noindent\textbf{Conclusions (for Chapter 1).} Brief paragraph summarizing the main theoretical takeaways.

% ---- Figure example (title under figure; centered) ----
\begin{figure}[h]
  \centering
  \includegraphics[width=0.7\linewidth]{b_chapters/chapter1/assets/isma_logo.png}
  \figsource{Authors’ created image}
  \caption{Example grouped illustration of motivation drivers - some dummy caption}
  \label{fig:motivation-drivers}
\end{figure}

% ---- Table example (title above; single spacing) ----
{\singlespacing
\begin{table}[h]
  \caption{Research results (example dummy caption)}
  \label{tab:research-results}
  \centering
  \begin{tabular}{lrrrr}
    \toprule
    Factor & A & B & C & D\\
    \midrule
    Pay & 2 & 23 & 23 & 3\\
    Recognition & 4 & 18 & 19 & 2\\
    Growth & 5 & 21 & 16 & 4\\
    \bottomrule
  \end{tabular}

  \vspace{2mm}
  \emph{Source:} author’s calculation based on data from Enterprise X.


  Example of citation \cite{porter2008}.
\end{table}
}
