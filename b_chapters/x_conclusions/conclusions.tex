\chapter{Conclusions}

The \textbf{Conclusions} chapter must clearly demonstrate that the aim and tasks of the Bachelor Paper have been fulfilled. This chapter is not a repetition of earlier text, but a synthesized summary that links the research aim, tasks, hypothesis, and results.

When preparing this section, students should ensure that:

\begin{itemize}
  \item \textbf{Confirmation of tasks.} Explicitly state that all tasks set out in the Introduction have been completed. A good practice is to connect each task with the chapter where it was accomplished.  
  Example:  
  “Task 1 (literature review) was completed in Chapter 1, providing a theoretical framework for the study. Task 2 (analysis of the case/system) was carried out in Chapter 2. Task 3 (empirical study) was presented in Chapter 3. Task 4 (development of proposals) was fulfilled in Chapter 4.”
  
  \item \textbf{Validation of hypothesis.} Clearly indicate whether the research hypothesis (if formulated) was validated or rejected, based on the obtained results.  
  Example:  
  “The proposed hypothesis was confirmed, as the empirical results demonstrated …”
  
  \item \textbf{Achievement of the aim.} Explicitly state that the aim of the thesis has been reached, supported by the fact that all tasks were accomplished and the hypothesis was addressed.  
  Example:  
  “Therefore, it can be concluded that the aim of the thesis has been achieved.”
  
  \item \textbf{Avoid repetition.} Do not copy entire findings from chapters. Instead, summarize them concisely to show the logical progression from tasks → results → aim.
  
  \item \textbf{Practical and theoretical value.} Conclude with a short statement on the significance of the results (e.g., how the findings may be used in practice or contribute to the field).
\end{itemize}

\noindent
\textbf{Recommended final formulation:}
“All tasks set out in the Introduction were accomplished (Tasks 1–4 correspond to Chapters 1–4). The research hypothesis was validated, and the aim of the thesis has been reached.”
