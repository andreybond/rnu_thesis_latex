\chapter{Case / System Analysis (Example Chapter~2)}
\label{chap:analysis}

% --- Guidelines for students ---
% The Bachelor thesis usually consists of 2–3 main chapters,
% but the exact structure is flexible.
% Use your judgement:
% - If two chapters are too short (<4–5 pages), consider merging them.
% - If a single chapter grows too large (>20–25 pages), consider splitting it.
% The goal is balance and readability, not a fixed number.

\section{General Characteristics of the Object / Case}
Provide a concise description of the object of study (enterprise, dataset, system, community, etc.). Explain context, history, ownership, or design aspects that are relevant for your research.

\section{Analysis of Internal and External Environment}
Use appropriate analytical frameworks (examples: SWOT, PEST, Five Forces, system diagrams, benchmarking). Present the logic of your analysis step by step.

% --- Example figure (replace with real content) ---
\begin{figure}[h]
  \centering
  \includegraphics[width=0.65\linewidth]{b_chapters/chapter1/assets/isma_logo.png}
  \figsource{Replace with your own diagram/model.}
  \caption{Example analytical framework diagram.}
  \label{fig:analysis-framework}
\end{figure}

\section{Data and Methods Used for the Analysis}
Briefly explain what data you are using here (statistics, survey, system logs, documents, etc.). Indicate how the data was collected, and what methods you are applying (e.g., descriptive statistics, content analysis, modelling).

% --- Example table (replace with real data) ---
{\singlespacing
\begin{table}[h]
  \caption{Example analysis results (dummy data)}
  \label{tab:analysis-results}
  \centering
  \begin{tabular}{lrr}
    \toprule
    Indicator & Value A & Value B\\
    \midrule
    Sample size         & 120 & -- \\
    Turnover rate (\%)  & 14.2 & 11.3\\
    Satisfaction index  & 3.8 & 4.2\\
    \bottomrule
  \end{tabular}
\end{table}
}

\section{Sub-Conclusions (for Chapter~2)}
End this chapter with a short subsection that summarizes the main analytical insights. Keep it focused: 1–2 paragraphs that motivate the next (empirical or practical) chapter.