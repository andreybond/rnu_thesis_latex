\chapter{Discussion, Limitations, and Future Work (Optional)}
\label{chap:optional-discussion}

% --- IMPORTANT FOR STUDENTS ---
% This chapter is FULLY OPTIONAL.
% Your thesis is complete without it, as long as you write
% a proper Conclusions chapter (see conclusions.tex).
%
% Only use this chapter if you feel it adds value
% beyond the mandatory Conclusions.
%
% If you decide not to include it, simply:
%   1. Delete this file and the folder b_chapters/chapter5/.
%   2. Remove or comment out the corresponding line
%      \chapter{Discussion}
Relate findings to literature; limitations; future work directions.

%      in main.tex.

\section{Extended Discussion (Optional)}
Here you may reflect on your findings in greater depth than in the Conclusions.  
Examples:
\begin{itemize}
  \item Compare your results to other studies in the field.
  \item Highlight unexpected findings or patterns.
  \item Discuss theoretical implications that go beyond your specific case.
\end{itemize}

\section{Limitations of the Study (Optional)}
State openly the constraints of your research:
sample size, data quality, scope, time/resources, methodological choices.  
Be honest but concise — acknowledging limitations strengthens credibility.

\section{Future Work and Recommendations (Optional)}
Suggest what could be done after your thesis:
\begin{itemize}
  \item New research questions your work raises.
  \item Practical applications, improvements, or prototypes to be developed.
  \item Follow-up studies, experiments, or datasets worth exploring.
\end{itemize}

\section*{Final Note}
Remember: the \textbf{Conclusions chapter (mandatory)} must already
confirm tasks, validate the hypothesis, and state that the aim is reached.  
This optional chapter is only for broader reflection and is not required.
