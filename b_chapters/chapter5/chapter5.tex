\chapter{Discussion, Limitations, and Future Work (Optional)}
\label{chap:optional-discussion}

% --- IMPORTANT FOR STUDENTS ---
% This chapter is FULLY OPTIONAL.
% Your thesis is complete without it, as long as you write
% a proper Conclusions chapter (see conclusions.tex).
%
% Only use this chapter if you feel it adds value
% beyond the mandatory Conclusions.
%
% If you decide not to include it, simply:
%   1. Delete this file and the folder b_chapters/chapter5/.
%   2. Remove or comment out the corresponding line
%      \chapter{Discussion, Limitations, and Future Work (Optional Example Chapter~5)}
\label{chap:optional-discussion}

% --- IMPORTANT NOTE FOR STUDENTS ---
% This chapter is fully optional.
% Your thesis is complete without it, as long as you have a proper Conclusions chapter.
% Use it only if you feel that broader reflection or forward-looking remarks
% add significant value to your work (and your supervisor approves).

This chapter is meant as an *extra space* for broader reflection.  
It can be useful if you want to go beyond the mandatory Conclusions and show awareness of the bigger picture.  
However, it should remain concise and focused — do not just repeat results already covered in previous chapters.

\section{Extended Discussion (Optional)}
\label{sec:extended-discussion}
Here you may reflect on your findings in greater depth.  
Examples of what to include:

\begin{itemize}[leftmargin=1.2cm]
  \item Compare your results to other studies or real-world applications.  
  \item Highlight unexpected outcomes or open questions.  
  \item Discuss theoretical or practical implications that extend beyond your specific project.  
\end{itemize}

\paragraph{Tip.} This section can be especially valuable if your work connects to ongoing research, industrial practice, or societal issues.

\section{Limitations of the Study (Optional)}
\label{sec:limitations}
Every project has constraints — being open about them increases credibility.  
Possible categories:

\begin{itemize}[leftmargin=1.2cm]
  \item \textbf{Scope.} Narrow focus, specific use case, or limited generalizability.  
  \item \textbf{Data/technical.} Small datasets, noisy logs, hardware/software constraints.  
  \item \textbf{Methodological.} Simplifying assumptions, lack of longitudinal data, limited metrics.  
\end{itemize}

\paragraph{Note.} Do not exaggerate — just acknowledge realistically what was and wasn’t possible within your timeframe and resources.

\section{Future Work and Recommendations (Optional)}
\label{sec:future}
Suggest what could be done after your thesis.  
Examples:

\begin{itemize}[leftmargin=1.2cm]
  \item Follow-up studies or new research questions raised by your findings.  
  \item Practical improvements to your system, prototype, or methods.  
  \item Opportunities for scaling up, applying to other domains, or testing with different datasets.  
\end{itemize}

This section is not required, but it can demonstrate initiative and forward thinking.

\section*{Final Note}
\label{sec:finalnote}
Remember: the \textbf{mandatory Conclusions chapter} must already confirm the research tasks, validate the hypothesis, and state whether the aim of the thesis has been achieved.  
This optional chapter is only for broader reflection and forward-looking remarks — use it only if it adds real value.

%      in main.tex.

\section{Extended Discussion (Optional)}
Here you may reflect on your findings in greater depth than in the Conclusions.  
Examples:
\begin{itemize}
  \item Compare your results to other studies in the field.
  \item Highlight unexpected findings or patterns.
  \item Discuss theoretical implications that go beyond your specific case.
\end{itemize}

\section{Limitations of the Study (Optional)}
State openly the constraints of your research:
sample size, data quality, scope, time/resources, methodological choices.  
Be honest but concise — acknowledging limitations strengthens credibility.

\section{Future Work and Recommendations (Optional)}
Suggest what could be done after your thesis:
\begin{itemize}
  \item New research questions your work raises.
  \item Practical applications, improvements, or prototypes to be developed.
  \item Follow-up studies, experiments, or datasets worth exploring.
\end{itemize}

\section*{Final Note}
Remember: the \textbf{Conclusions chapter (mandatory)} must already
confirm tasks, validate the hypothesis, and state that the aim is reached.  
This optional chapter is only for broader reflection and is not required.
